\chapter*{Introduction}
\addcontentsline{toc}{chapter}{Introduction}

In the modern-era Internet most web pages operate for profit. Most of them, however, choose to provide free content
in exchange for displaying paid advertisements, or ads for short. Unfortunately, not all websites play fair. Some of them
concentrate on displaying as many ads as possible, and generate traffic by using click-baits 
and other shady practices. Even websites with valuable content can have an overwhelming amount 
of ads. This leads to grave dissatisfaction of some portion of users. To make their browsing experience better,
they turn to ad-blocking extensions.

Operation of the most popular adblockers is based on user-curated lists. They have been around for such a long
time that they became really effective. Not only they are able to block the vast majority of advertisements
but also they rearrange the remaining content to provide seamless experience.
In order to circumvent them, one has to put an ongoing effort to be able
to catch up. It is such a common occurrence that even some browsers have ad-blocking
functionality built-in (e.g. Opera).

The amount of users utilizing ad-blockers cannot be ignored. In 2019 there was over 615 million devices worldwide
with adblocker installed \cite{pagefair:adblock-report}. This, in turn, leads to loss of revenue for many businesses.
To combat this, some of them choose to deploy anti-adblockers, i.e., scripts that detect adblocking extensions
and react in some way. They can generate a visible warnings or even even block the website's content entirely.

The anti-adblockers come in many different fashions. Some of them are simple, custom-made scripts that 
set up an obvious bait and check for reaction typical to adblocking extensions.
They may check if some file was not loaded or if the advertisement banners have been tampered with.
Some solutions perform more than one check. The most advanced ones also employ mechanisms that
make the analysis and reverse-engineering harder. Examples include code minification
or obfuscation involving \emph{eval} function.

Some ad providers, publishers and extension creators recognize users' discontent 
and start initiatives like Acceptable Ads \cite{acceptableads}.
The idea is to have a white list of ads that meet certain criteria such as non-intrusiveness.
Such list is embedded into ad-blocking extensions and enabled by default.
Others just focus on making sure that the ads are displayed and continue to bring profit.
The way to do that is by constantly improving anti-adblocking solutions to stay ahead.

The people behind adblockers have mixed approaches to anti-adblocking warnings.
Some of them choose to let all of them be, others only if they are dismissible (and thus 
less effective from the point of view of the site owners). Naturally, there are also other 
players who develop extensions which block all such warnings, regardless of their
intrusiveness. The whole conflict of interests, opinions and values leads to an arms race.

A regular study of anti-adblocking scripts can help both sides of the barricade. Knowing how such scripts behave
can help create better methods of detecting them. That, in turn, can lead to creation of better blocking tools.
On the other hand, being able to automatically detect some scripts usually means that it is also possible
to block them. As a consequence, studying different detection methods can lead to better anti-adblockers as well.

In 2018 Zhu et al. \cite{DBLP:conf/ndss/ZhuHQSY18} proposed a sophisticated method 
of automated detection of such scripts. This method uses what is called a Differention Execution Analysis. 
The whole approach consists of several steps.
First, at least two JavaScript execution traces have to be collected. An execution trace is a list of all events such
as function enter or leave that occurred during program execution. The first trace is of website's code executed
in an environment without any adblocking extensions. The second one is collected with such a plugin active.
These two traces are later compared and checked if there are any differences between them.
If there are, they can be attributed to anti-adblocking scripts.

There are lots of difficulties to get the whole mechanism working. First of all, trace collection has
to be hand-made. There is no off-the-shelf solution that would meet all requirements of this method.
Second, JavaScript event-based execution model makes it problematic to compare execution traces
as the order and number of events can differ greatly. This issue is solved 
by processes called trace slicing (or untangling) and trace matching. The former is a method of gathering 
execution events into subtraces that correspond to different events. The latter is a process of
pairing subtraces from two different execution traces.
Third, a website can incorporate tons of noise. The web page can utilize some random functions, 
there can be network errors, the content can be generated or selected dynamically.
The authors of the method give a glimpse of how to battle each of the difficulties, 
but leave out a lot of details in most cases. 

\todo{transform into full sentences?}
This work is about implementing a similar system based on the aforementioned idea.
The contributions of the thesis are the following:
\begin{itemize}
  \item The entire method has been implemented \todo{explain what end-to-end means} 
           end-to-end and each step explained in detail.
  \item A novel approach based on Stable Marriage Problem to solve trace matching problem
           has been introduced.
  \item A new approach to filtering execution noises has been introduced and tested.
  \item The entire pipeline has been evaluated and used to conduct a small-scale study on the most
           popular websites in Poland.
  \item An overview of anti-adblocking methods has been presented, based on the pipeline evaluation.
\end{itemize}

First part of the system is Chromium browser with manually instrumented V8 engine that produces
execution traces. The browser is run automatically a few times for each tested website.
Later, the traces are filtered and analyzed to detect execution differences.

The implementation has been manually tested on 100 websites and proved to be a useful tool 
in detecting and analyzing anti-adblocking scripts. 
The analysis of those websites resulted in identification of the most popular mechanisms and solutions.

The study conducted on 300 most popular web pages in Poland revealed that anti-adblocking scripts
are far more widespread than what can be conjectured by just studying visible reactions.
