\chapter*{Introduction}
\addcontentsline{toc}{chapter}{Introduction}

In modern-era Internet most webpages operate for profit. Most of them, however, choose to provide free content
in exchange for displaying paid ads. Unfortunately, not all websites play fair. Some of them
concentrate on displaying as many ads as possible and generate traffic by using click-baits 
and other shady practices. Even websites with valuable content can have overwhelming amount 
of ads. This leads to grave dissatisfaction of some portion of users. To make their browsing experience better,
they turn to use of ad-blocking extensions.

The most popular adblockers operate based on user-curated lists. They have been around for such a long
time that they are really effective. In order to circumvent them, one has to put an ongoing effort to be able
to catch up. It is such a common occurence that even some browsers have built-in ad-blocking
functionality (e.g. Opera).

The amount of users utilizing ad-blockers cannot be ignored. In 2019 there was over 615 million devices worldwide
with adblocker installed \cite{pagefair:adblock-report}. This, in turn, leads to loss of revenue for many businesses.
To combat this, some of them choose to deploy anti-adblockers, i.e. scripts that detect adblocking extensions
and react in some way. They can generate a visible warnings or even even block the website's content entirely.

The anti-adblockers come in many different fashions. Some of them are simple, custom-made scripts that 
set up an obvious bait and check if there was a reaction typical to adblocking extension (e.g. a file names \emph{ads.js}
has not been loaded). There are also more effective solutions that use a number of tricks to make the analysis
and counteraction harder.

Some ad providers and publishers recognize users' discontent and start initiatives like Acceptable Ads \cite{acceptableads}.
The idea is to whitelist ads that meet certain criteria (mostly non-intrusiveness).
Others care less about users and just improve the anti-adblocking solutions 
to make sure that ads continue to bring profit.

The people behind adblockers have mixed approaches to anti-adblocking warnings.
Some of them choose to let all of them be, others only if they are dismissible (and thus 
less effective from the point of view of the site owners). Naturally, there are also other 
players who develop extensions which block all such warnings, regardless of their
intrusiveness. This leads to an arms race.

A regular study of anti-adblocking scripts can help both sides of the barricade. Knowing how such scripts behave
can help create better methods of detecting them. That, in turn, can lead to creation of better blocking tools.
On the other hand, being able to automatically detect some scripts usually means that it is also possible
to block them. As a consequence, studying different solution and detection methods can lead to better anti-adblockers as well.

In 2018 Zhu et al. \cite{DBLP:conf/ndss/ZhuHQSY18} proposed a sophisticated method of automatically detecting such
scripts. This method uses what is called a Differention Execution Analysis. The whole approach consists of several steps.
First, at least two JavaScript execution traces have to be collected. One being the website's code executed
in an environment without any adblocking extensions and second with such plugin active.
Those two traces are later compared and checked if there are any differences between them.
If there are, they can be attributed to anti-adblocking scripts.

There are lots of difficulties in getting the whole mechanism to work. First of all, trace collection has
to be hand-made. There is no off-the-shelf solution that would meet the requirements of this method.
Second, JavaScript event-based execution model makes it problematic to compare execution traces
as the order and number of events can greatly differ. This issue is solved 
by processes called trace slicing (or untangling) and trace matching. The former is a method of gathering 
execution events into subtraces that correspond to different events. The latter is a proccess of
pairing subtraces from two different execution traces.
Third, a website can incorporate tons of noise. The webpage can utilize some Random Number Generators, 
there can be network errors, the content can be generated or selected dynamically.

The authors of the method give a glimpse of how to battle each of the difficulties, 
but leave out a lot of details in most cases. The contibution of this thesis are the following:
\begin{itemize}
  \item The entire method has been implemented end-to-end and each step explained in detail.
  \item A novel approach based on Stable Marriage Problem to solve trace matching problem
           has been introduced.
  \item A new approach to filtering execution noises has been introduced and tested.
  \item The entire pipeline has been evaluated and used to conduct a small-scale study on the most
           popular websites in Poland.
  \item An overview of anti-adblocking methods has been presented, based on the pipeline evaluation.
\end{itemize}

The implementation has been manually tested on 100 websites and proved to be a useful tool 
in detecting and analysing anti-adblocking scripts. The analysis of those websites
resulted in identification of the most popular mechanisms and solutions.

The study conducted on 300 most popular webpages in Poland revealed that anti-adblocking scripts
are far more widespread than what can be conjectured by just studying visible reactions.
