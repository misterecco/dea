%
% Niniejszy plik stanowi przykład formatowania pracy magisterskiej na
% Wydziale MIM UW.  Szkielet użytych poleceń można wykorzystywać do
% woli, np. formatujac wlasna prace.
%
% Zawartosc merytoryczna stanowi oryginalnosiagniecie
% naukowosciowe Marcina Wolinskiego.  Wszelkie prawa zastrzeżone.
%
% Copyright (c) 2001 by Marcin Woliński <M.Wolinski@gust.org.pl>
% Poprawki spowodowane zmianami przepisów - Marcin Szczuka, 1.10.2004
% Poprawki spowodowane zmianami przepisow i ujednolicenie 
% - Seweryn Karłowicz, 05.05.2006
% Dodanie wielu autorów i tłumaczenia na angielski - Kuba Pochrybniak, 29.11.2016

% dodaj opcję [licencjacka] dla pracy licencjackiej
% dodaj opcję [en] dla wersji angielskiej (mogą być obie: [licencjacka,en])
\documentclass[en]{pracamgr}

% Dane magistranta:
\autor{Tomasz Kępa}{359746}

% Dane magistrantów:
%\autor{Autor Zerowy}{342007}
%\autori{Autor Pierwszy}{342013}
%\autorii{Drugi Autor-Z-Rzędu}{231023}
%\autoriii{Trzeci z Autorów}{777321}
%\autoriv{Autor nr Cztery}{432145}
%\autorv{Autor nr Pięć}{342011}

\title{Detecting Anti-Adblockers using Differential Execution Analysis}
\titlepl{Wykrywanie skryptów blokujących rozszerzenia typu AdBlock w przeglądarkach}

%\tytulang{An implementation of a difference blabalizer based on the theory of $\sigma$ -- $\rho$ phetors}

%kierunek: 
% - matematyka, informacyka, ...
% - Mathematics, Computer Science, ...
\kierunek{Computer Science}

% informatyka - nie okreslamy zakresu (opcja zakomentowana)
% matematyka - zakres moze pozostac nieokreslony,
% a jesli ma byc okreslony dla pracy mgr,
% to przyjmuje jedna z wartosci:
% {metod matematycznych w finansach}
% {metod matematycznych w ubezpieczeniach}
% {matematyki stosowanej}
% {nauczania matematyki}
% Dla pracy licencjackiej mamy natomiast
% mozliwosc wpisania takiej wartosci zakresu:
% {Jednoczesnych Studiow Ekonomiczno--Matematycznych}

% \zakres{Tu wpisac, jesli trzeba, jedna z opcji podanych wyzej}

% Praca wykonana pod kierunkiem:
% (podać tytuł/stopień imię i nazwisko opiekuna
% Instytut
% ew. Wydział ew. Uczelnia (jeżeli nie MIM UW))
\opiekun{dr Konrad Durnoga\\
 Institute of Informatics
  }

% miesiąc i~rok:
\date{August 2019}

%Podać dziedzinę wg klasyfikacji Socrates-Erasmus:
\dziedzina{ 
%11.0 Matematyka, Informatyka:\\ 
%11.1 Matematyka\\ 
11.3 Informatics, Computer Science
%11.3 Informatyka\\ 
%11.4 Sztuczna inteligencja\\ 
%11.5 Nauki aktuarialne\\
%11.9 Inne nauki matematyczne i informatyczne
}

%Klasyfikacja tematyczna wedlug AMS (matematyka) lub ACM (informatyka)
\klasyfikacja{Software and its engineering.~Dynamic analysis}

% Słowa kluczowe:
\keywords{dynamic analysis, differential execution analysis, javascript, anti-adblockers, ads}

% Tu jest dobre miejsce na Twoje własne makra i~środowiska:
\newtheorem{defi}{Definicja}[section]

\usepackage{xcolor}
\newcommand{\todo}[1]{\colorbox{yellow}{ \color{red} \textbf{TODO}: {#1}}}

% koniec definicji

\begin{document}
\maketitle

%tu idzie streszczenie na strone poczatkowa
\begin{abstract}
Ads are the main source of income of numerous websites. 
However, some of them are fairly annoying which causes many users to use adblocking browser extensions. 
Some services, in turn, use specialized scripts to detect such plug-ins 
and silently report them or block some content as a punishment. 
The goal of this thesis is to build a pipeline for detecting such scripts based on a differential execution analysis, 
a method provided by other authors in 2018. 
Such a mechanism can be used later to analyze the prevalence of anti-adblockers 
on Polish websites or to build an extension capable of circumventing such scripts.

\end{abstract}

\tableofcontents
%\listoffigures
%\listoftables

\chapter*{Introduction}
\addcontentsline{toc}{chapter}{Introduction}

\todo{}

\chapter{Basic concepts}
\section{Definitions}

\begin{itemize}
  \item Execution event -- each occurence of control executing some statement, entering or leaving a control structure etc.
  \item Execution trace -- a series of execution events collected during program execution. 
           It is dependent both on program structure and it's input (also implicit such as randomly generated numbers)
  \item Execution index
  \item Stable marriage problem
\end{itemize}

\section{JavaScript execution model}

\chapter{Related work}
\section{Zhu et. al}

\chapter{Trace collection}
\section{JavaScript rewriting}
\section{JavaScript engine instrumentation}

\chapter{Trace collection by V8 instrumentation}
\section{V8 architecture}
\section{V8 usage in chromium}
\section{Chrome's extensions architecture}
\section{V8's \emph{-{}-trace} flag}
\section{Bytecode injection}
\section{Controlling Chrome programatically}

\chapter{Trace analysis}
\section{Trace untangling using execution index}
\subsection{Optimizations}
\section{Trace matching using SMP}
\section{Noise filtering}

\chapter{Evaluation}

\begin{thebibliography}{99}
\addcontentsline{toc}{chapter}{Bibliography}

\bibitem[Bea65]{beaman} Juliusz Beaman, \textit{Morbidity of the Jolly
    function}, Mathematica Absurdica, 117 (1965) 338--9.

\bibitem[Blar16]{eb1} Elizjusz Blarbarucki, \textit{O pewnych
    aspektach pewnych aspektów}, Astrolog Polski, Zeszyt 16, Warszawa
  1916.

\bibitem[Fif00]{ffgg} Filigran Fifak, Gizbert Gryzogrzechotalski,
  \textit{O blabalii fetorycznej}, Materiały Konferencji Euroblabal
  2000.

\bibitem[Fif01]{ff-sr} Filigran Fifak, \textit{O fetorach
    $\sigma$-$\rho$}, Acta Fetorica, 2001.

\bibitem[Głomb04]{grglo} Gryzybór Głombaski, \textit{Parazytonikacja
    blabiczna fetorów --- nowa teoria wszystkiego}, Warszawa 1904.

\bibitem[Hopp96]{hopp} Claude Hopper, \textit{On some $\Pi$-hedral
    surfaces in quasi-quasi space}, Omnius University Press, 1996.

\bibitem[Leuk00]{leuk} Lechoslav Leukocyt, \textit{Oval mappings ab ovo},
  Materiały Białostockiej Konferencji Hodowców Drobiu, 2000.

\bibitem[Rozk93]{JR} Josip A.~Rozkosza, \textit{O pewnych własnościach
    pewnych funkcji}, Północnopomorski Dziennik Matematyczny 63491
  (1993).

\bibitem[Spy59]{spyrpt} Mrowclaw Spyrpt, \textit{A matrix is a matrix
    is a matrix}, Mat. Zburp., 91 (1959) 28--35.

\bibitem[Sri64]{srinis} Rajagopalachari Sriniswamiramanathan,
  \textit{Some expansions on the Flausgloten Theorem on locally
    congested lutches}, J. Math.  Soc., North Bombay, 13 (1964) 72--6.

\bibitem[Whi25]{russell} Alfred N. Whitehead, Bertrand Russell,
  \textit{Principia Mathematica}, Cambridge University Press, 1925.

\bibitem[Zen69]{heu} Zenon Zenon, \textit{Użyteczne heurystyki
    w~blabalizie}, Młody Technik, nr~11, 1969.

\end{thebibliography}

\end{document}


%%% Local Variables:
%%% mode: latex
%%% TeX-master: t
%%% coding: latin-2
%%% End:
