\chapter{Evaluation}
\label{evaluation}

\section{The pipeline}

The goal of this work was to make most steps of the anti-adblocking analysis automatic.
Ideally, the system should be able to read a list of websites, collect traces and analyze
each of them. At the end a meaningful result should be returned.

The entire system is brought together by a program written in Python.
It is controlled by a extensive set of flags.
The pipeline is the following:
\begin{itemize}
  \item A website to analyze is read as an argument or from a file, when special flag is provided.
  \item If selected, 3 positive and 3 negative traces are collected for the website (before each browser run,
           all cookies are deleted). After each all unwanted traces are deleted (extensions, iframes).
  \item If selected, the analysis is run, the diff is saved in a selected location.
  \item Optionally, the traces are deleted.
  \item Each step is logged.
\end{itemize}

\todo{Maybe change that to just 0, >0}
The system does not provide an explicit answer whether a website utilizes an anti-adblocking script.
The result of an analysis is the diff that may be of great help when answering that question.

\section{Methodology}
\label{methodology}

The methodology is the same as in the original article \cite{DBLP:conf/ndss/ZhuHQSY18}.
First, the effectiveness of the method is tested on a sets of negative and positive examples.
Then, the system is used to conduct a larger study.

One important difference from the original approach is that, unless the page has a visible anti-adblocking
reaction on the main page, the pages are always tested on some subpage.

The original authors composed negative examples set from 100 websites that do not display ads,
The reasoning behind this is that they are guaranteed not to contain anti-adblocking scripts.
However, such choice of examples as a benchmark seems flawed.
One of the key challenges in this method is filtering of noise. If there are no ads, 
then there are practically no sources of execution differences between runs with and without an adblocker.
A better choice of websites to evaluate on would be a mix of websites without ads and sites that
display ads, but are known not to contain anti-adblockers.
Unfortunately, finding such websites is not easy as it entails a detailed analysis of the entire code
served by each site. Certainly not a task for one person to be done in a reasonable timespan.
For this reason, the negative examples are only sites that do not serve ads. There is 50 of them.

The original authors had an extensive list of websites with anti-adblockers collected in their
previous studies (almost 700 positions). Unfortunately, they do not share that list. 
For this reason, a similar list of positive examples had to be created from scratch.
It is a mix of websites encountered during daily browsing and sites appearing in the
Polish anti-adblock list \cite{github:anti-adblock-list}.
To avoid time-consuming code analysis at the stage of composing the list, 
all selected websites contain visible anti-adblock warnings.
Other important note -- the list contain at most handful of websites from Alexa 300 list.

The adblocking extension have different policies when it comes to their official, default filters.
AdBlock Plus behaves politely and respects publishers' right to ask users to show some support
and disable the extension, whatever form it takes \cite{adblock:policy}.
On the other hand, uBlock Origin employs a different rule. If the warning is not dismissible or is annoying,
it will be blocked \cite{vice:ublock-policy}.

Since usage of AdBlock Plus may expose a wider variety of adblock walls and warnings, it was chosen for the tests.

\section{Negative examples}

The list of websites without ads, on which the system was tested is presented in the table \ref{table:negative}.
The number of execution differences found is listed in the last column.

\csvreader[webisteList,
  longtable= r | p{13cm} | r ,
  table head=\caption{Results of system evaluation on negative examples -- pages without ads}
  \label{table:negative}\\
  \hline & URL & \thead{Diff. \\ count}\\\hline
]{tsv/negative.tsv}{}%
{\thecsvrow & \url{\adr} & \cnt}%

Ideally, there should be no execution differences on any website listed here. Nonetheless, a few differences were identified (positions 44-50).
Let us have a look why.

\subsubsection{\url{kinoelektronik.pl}}
The origin of difference is unclear. There does not seem to be any element
blocked by AdBlock. Also, there are no blocked network requests. But for some reason there are differences in
jQuery traversals.

\subsubsection{\url{ethz.ch}}
The cause of the difference is the tracking code.

\subsubsection{\url{szkolaimpro.pl}}
The difference is due to the progress bar.

\subsubsection{\url{www.paperswithcode.com}}
Execution divergence stems from Sentry -- an error reporting solution.

\subsubsection{\url{www.whitehouse.gov}}
The origin of the difference is unclear.

\subsubsection{\url{www.worldanimalprotection.org}}
Similar case to \url{kinoelektronik.pl}. The difference is reflected in jQuery traversals,
but it is unknown why the DOM trees are different

\subsubsection{\url{zoo.waw.pl}}
The same case as \url{kinoelektronik.pl} and \url{www.worldanimalprotection.org}.


\section{Positive examples}

The list of websites with visible anti-adblock reaction is presented in the table \ref{table:positive}.
\todo{reword}
Each row consists of the URL, number of traces with execution differences, relevance of the traces
and anti-adblocker type.

\csvreader[webisteList,
  longtable= r | p{8cm} | r | c | p{4cm} ,
  table head=\caption{Results of system evaluation on positive examples -- pages with adblock walls}
  \label{table:positive}\\
  \hline & URL & \thead{Diff. \\ count} & \thead{Diff \\ rel.?} & Type\\\hline
]{tsv/positive.tsv}{}%
{\thecsvrow & \url{\adr} & \cnt & \res & \type}%

In most cases the system is able to find execution differences, which is a very good sign.
In the following sections we will analyze when the system fails and discuss
a variety of solutions.

The analysis of 50 websites incorporating adblock walls allows us to create a simple
classification of anti-adblockers:

\begin{itemize}
  \item Mechanisms
    \begin{itemize}
      \item With a bait
        \begin{itemize}
         \item File as a bait
         \begin{itemize}
           \item Variable value is changed
           \item An element injected into DOM
           \item Error handler on network request
         \end{itemize}
           \item DOM element as a bait
        \end{itemize}
      \item Real ads visibility checks
      \item Real ads load check
    \end{itemize}
  \item Solutions:
    \begin{itemize}
      \item Custom scripts
      \item Off the shelf modules:
        \begin{itemize}
          \item BlockAdBlock \cite{github:blockadblock}
          \item Adblock Detect \cite{adblock-detect}
          \item Kill AdBlock \cite{kill-adblock}
          \item Admiral's Recover \cite{admiral:recover}
          \item \emph{an\_message\_display}
          \item \emph{Adb\_Detector}
        \end{itemize}
    \end{itemize}
\end{itemize}

The most popular solutions are custom scripts and open-source BlockAdBlock (already mentioned in \ref{anti-adblockers}).
Other open-source scripts include Adblock Detect and KillAdBlock. One website also used a paid solution -- Admiral platform.
The origin of the last two solutions is unclear, but the code was clearly the same (the names listed are 
distictive functions names)

Some custom scripts can be really wide-spread -- some publishers own a huge number of domains and it makes 
sense to create one robust solution and use it in every website. One example is Wirtualna Polska group (positions 25, 27), which shares
a framework that includes a custom anti-adblocking code.

Another important note -- in most cases some of the differences found originated in third-party ads
or tracking code (Google Analytics, AdWords, Gemius trackers, Twitter buttons, Facebook buttons etc.)

\subsection{Websites with unsatisfactory results}

\subsubsection{\url{jakimasklad.pl}}
Brief analysis of the code shows that the site incorporates BlockAdBlock \cite{github:blockadblock}. 
Turns out that, is this case, the module's obfuscation if effective enough to circumvent our system.
It is interesting that a number of different websites uses the same solution and the system
was able to pinpoint the differences in those cases.

\subsubsection{\url{www.srebrnestawy.pl}}
In this case the code that activates the warning seems very simple and there was no reason why
it should not be detected by the system. Listing \ref{js:srebrnestawy} shows the code.

\lstinputlisting[language=JavaScript, caption=Anti-adblocking script of \url{www.srebrnestawy.pl}, 
                       label=js:srebrnestawy]{js/srebrnestawy.js}

The reason why it was not detected is that the bait was based on file \emph{advert.js}, which is not loaded 
even when AdBlock is disabled, due to an HTTP 404 error. Consequently, the warning is displayed even 
when AdBlock is disabled and the system was right -- there were no execution differences.

\subsubsection{\url{www.cyberciti.biz}}
In this case the difference found corresponds to PageAd reaction (it is good that is was found)
but the check that activates the adblock warnings was not found.

\todo{Don't refer audiklub or refer it more precisely}
Anti-adblocker used on this page works almost exactly the same as the one presented in section \ref{audiklub}.
This site, however, uses Cloudflare's Rocket Loader \cite{cloud-flare:rocket-loader}.

Rocket Loader is a powerful mechanism developed by Cloudflare that is able to cache JavaScript code and defer its execution 
until the entire webpage is rendered. The way it works is the following: 
\begin{itemize}
  \item  Before serving the page, the server inspects the website and changes all JS scripts 
            to some type not identified by the browser as JavaScript. 
            The type also has  to be unique so that those scripts are easy to find later.
  \item The server also encloses a special JavaScript module that will activate when the entire webpage is loaded.
  \item JavaScript module finds all scripts with changed type and executes them.
\end{itemize}

Since the module that executes the scripts is a JavaScript code, the system should still be able
to detect the anti-adblocking code. Unfortunately, it cannot and the reason why is unclear, although 
the most probable seems to be the noise filtering. It is a bit aggressive. In a case when some function
does many things at once, and it has to traverse the entire DOM and do something else later, 
it may be filtered out. DOM may be different if there is a dynamic content 
or ads that greatly change on each reload.
Rocker Loader has to find all scripts and execute them. Thus, it is a DOM traversal followed by executing
another code, which is a weak spot of the filtering technique.

\subsubsection{\url{torrentcity.pl}}
This is an example of rather sophisticated custom anti-adblocking script.
Listing \ref{js:torrentcity} shows the code that performs the check.

\lstinputlisting[language=JavaScript, caption=Anti-adblocking script of \url{torrentcity.pl}, 
                       label=js:torrentcity]{js/torrentcity.js}
                       
The script first sets up a bait by injecting a new script element into DOM. It is set up to download
a file named \emph{ads.js}, which should be blocked by any adblocker.

Next, the error handler is set up to activate the reaction. If the load succeeds, it still
makes sure that are are displayed before.

Functions called as event handlers or via \emph{setTimeout} are a weak spot of this method,
not just this exact implementation as nobody yet found a good way to automatically
analyze such unpaired events.

\subsubsection{\url{player.pl}}
An example of a really robust solution, probably used on every site belonging to TVN.
Listing \ref{js:tvn} shows the checks. The website utilized more than one check
to increase detection effectiveness.
The checks itself are similar to what we have already seen. First one creates an element as a bait 
and check if it is visible. The second uses different bait, a file, and reacts with in callbacks.
The system fails here because the script uses all the techniques that exploit its weak spots.

\lstinputlisting[language=JavaScript, caption=Anti-adblocking script of \url{player.pl}, 
                       label=js:tvn]{js/tvn.js}

\subsubsection{Other websites}
In case of other websites where relevant execution divergence was not found 
the reasons were the same -- either the check was intertwined between other activities
in a way that filtering would discard it or callbacks were utilized.

\subsection{Detected anti-adblockers}

This section presents the interesting methods that have not been covered yet.
Simple examples were preferred to avoid showing a cluttered code.

\subsubsection{\url{www.audiklub.pl}}
\label{audiklub}
This is the simplest anti-adblocking script one can think of. The website includes just one advertisement, 
but when adblocker
is detected, the content is blocked.

Listing \ref{js:audiklub} shows the check detected by the system.

\lstinputlisting[language=JavaScript, caption=Anti-adblocking script of \url{www.audiklub.pl}, 
                       label=js:audiklub]{js/audiklub.org.js}

First, the trap is set up. It is a file named \emph{ads.js}.
Its content is just one line, which sets \emph{canRunAds} variable to true.

The main check is performed in the \emph{window.onload} callback. The script simply checks
if \emph{canRunAds} variable has been set to true. If not, it replaces the content div with an anti-adblocker warning.

\subsubsection{\url{kursbootstrap.pl}}
The script on this website serves as an example of an anti-adblocker that uses file as a bait
but the check is performed by injecting an element into DOM instead of setting a global variable
(listing \ref{js:kursbootstrap}).

\lstinputlisting[language=JavaScript, caption=Anti-adblocking script of \url{kursbootstrap.pl}, 
                       label=js:kursbootstrap]{js/kursbootstrap.js}
                       
\subsubsection{\url{who-called.eu}}
The code from this website demonstrates how anti-adblocking script can detect if real
ads are displayed, without using bait (listing \ref{js:whocalled}).
The detecting function is run 5 seconds after the page is loaded. It checks if the content 
of the element displaying ads is empty (contains 0 non-whitespace characters).

\lstinputlisting[language=JavaScript, caption=Anti-adblocking script of \url{who-called.eu}, 
                       label=js:whocalled]{js/whocalled.js}
                       
\todo{BlockAdBlock + eval}

\todo{kimcartoon}

\todo{mention other off-the-shelf solutions}


\section{Small scale study}

\todo{rewrite}

The system has later been run on almost 300 top websites from Alexa 500 list (Poland only) \cite{alexa-list}.
The list of visited websites and results is attached in Appendix A. Some webistes were not visited because they
contain adult content, some because they were a Content Delivery Network or something similar. Those
websites, instead of URLs, are commented out (line starts with --).

Some results are marked with "TO" instead of a number of differences found. This is because
there is a limited time for each trace collection (30 seconds). After that the browser is closed.
In some cases the browser would not close by itself and it has to be killed. It is because 
some websites keep writing the trace and they write a lot of data (some can reach almost 50Gb in around 2 minutes).
In such cases the website is skipped.

The number of webistes, social media links and similar elements was not reported, but it seems that the more 
elements of this kind a website incorporate, the more likely it is that there will be anti-adblocking reaction of some kind.
It is especially true in case of third-party ads, which more often than not report adblocker usage.

Looking at the results, it seems that anti-adblocking scripts are widespread. 
Nevertheless, the system was not properly evaluated for false positives.
In fact, neither was the original authors' system, see section \ref{methodology}.
For this reason it would be advisable to verify the results and manually check at least 
a few dozens of positions from the list to make sure that the system is not producing 
a lot of false positives.


\section{Conclusion}

The system is certainly capable of discovering a variety of anti-adblocker solutions.
It seems, however, that the pareto principle \footnote{80\% of work is done in 20\% of time}
is relevant here and to increase the coverage and lower false positive rate
much more work and research has to be done.

When it comes to this system, after performing all the experiments, a few improvements come to mind:
\begin{itemize}
  \item The trace collection should have less overhead on the browser. For large scale study
           spending 10 minutes on each website to collect traces is too long.
           One idea is to, instead of constructing a strings map during analysis, have
           Chrome construct it. This, is conjunction with a use of some binary format, will
           result in much more compact files and less time wasted on writing to files.
  \item Although the filtering was really aggressive, in many cases there were still too many differences.
           Some additional method to decide which differences are relevant is needed.
\end{itemize}

