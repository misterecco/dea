%
% Niniejszy plik stanowi przykład formatowania pracy magisterskiej na
% Wydziale MIM UW.  Szkielet użytych poleceń można wykorzystywać do
% woli, np. formatujac wlasna prace.
%
% Zawartosc merytoryczna stanowi oryginalnosiagniecie
% naukowosciowe Marcina Wolinskiego.  Wszelkie prawa zastrzeżone.
%
% Copyright (c) 2001 by Marcin Woliński <M.Wolinski@gust.org.pl>
% Poprawki spowodowane zmianami przepisów - Marcin Szczuka, 1.10.2004
% Poprawki spowodowane zmianami przepisow i ujednolicenie 
% - Seweryn Karłowicz, 05.05.2006
% Dodanie wielu autorów i tłumaczenia na angielski - Kuba Pochrybniak, 29.11.2016

% dodaj opcję [licencjacka] dla pracy licencjackiej
% dodaj opcję [en] dla wersji angielskiej (mogą być obie: [licencjacka,en])
\documentclass[en]{pracamgr}

% Dane magistranta:
\autor{Tomasz Kępa}{359746}

% Dane magistrantów:
%\autor{Autor Zerowy}{342007}
%\autori{Autor Pierwszy}{342013}
%\autorii{Drugi Autor-Z-Rzędu}{231023}
%\autoriii{Trzeci z Autorów}{777321}
%\autoriv{Autor nr Cztery}{432145}
%\autorv{Autor nr Pięć}{342011}

\title{Detecting Anti-Adblockers using Differential Execution Analysis}
\titlepl{Wykrywanie skryptów blokujących rozszerzenia typu AdBlock w przeglądarkach}

%\tytulang{An implementation of a difference blabalizer based on the theory of $\sigma$ -- $\rho$ phetors}

%kierunek: 
% - matematyka, informacyka, ...
% - Mathematics, Computer Science, ...
\kierunek{Computer Science}

% informatyka - nie okreslamy zakresu (opcja zakomentowana)
% matematyka - zakres moze pozostac nieokreslony,
% a jesli ma byc okreslony dla pracy mgr,
% to przyjmuje jedna z wartosci:
% {metod matematycznych w finansach}
% {metod matematycznych w ubezpieczeniach}
% {matematyki stosowanej}
% {nauczania matematyki}
% Dla pracy licencjackiej mamy natomiast
% mozliwosc wpisania takiej wartosci zakresu:
% {Jednoczesnych Studiow Ekonomiczno--Matematycznych}

% \zakres{Tu wpisac, jesli trzeba, jedna z opcji podanych wyzej}

% Praca wykonana pod kierunkiem:
% (podać tytuł/stopień imię i nazwisko opiekuna
% Instytut
% ew. Wydział ew. Uczelnia (jeżeli nie MIM UW))
\opiekun{Dr Konrad Durnoga\\
 Institute of Informatics
}

% miesiąc i~rok:
\date{August 2019}

%Podać dziedzinę wg klasyfikacji Socrates-Erasmus:
\dziedzina{ 
%11.0 Matematyka, Informatyka:\\ 
%11.1 Matematyka\\ 
11.3 Informatics, Computer Science
%11.3 Informatyka\\ 
%11.4 Sztuczna inteligencja\\ 
%11.5 Nauki aktuarialne\\
%11.9 Inne nauki matematyczne i informatyczne
}

%Klasyfikacja tematyczna wedlug AMS (matematyka) lub ACM (informatyka)
\klasyfikacja{Software and its engineering.~Dynamic analysis}

% Słowa kluczowe:
\keywords{dynamic analysis, differential execution analysis, javascript, anti-adblockers, ads}

% Tu jest dobre miejsce na Twoje własne makra i~środowiska:

\usepackage{graphicx}
\usepackage{xcolor}
\newcommand{\intodo}[1]{\colorbox{yellow}{ \color{red} \textbf{TODO}: {#1}}}

% PROMOTOR PROMOTOR PROMOTOR PROMOTOR PROMOTOR PROMOTOR PROMOTOR PROMOTOR
%\usepackage[top=1in, bottom=1.25in, left=1.8in, right=1.8in,marginparsep=10pt,marginparwidth=100pt]{geometry} %usunac te linijke, ona jest tylko po to by moje notatki na marginesach się miesciły
\usepackage[shadow,color=black!15,textsize=scriptsize]{todonotes}
\newcommand{\kdtodo}[1]{\todo[color=red!40,bordercolor=red,size=\footnotesize]{\textbf{TODO:}#1}}
\newcommand{\kdintodo}[1]{\todo[inline,color=red!40,bordercolor=red,size=\footnotesize]{\textbf{TODO:}#1}}
% END PROMOTOR

\usepackage{cite}
\usepackage{url}
\makeatletter
\g@addto@macro{\UrlBreaks}{\UrlOrds}
\makeatother
\usepackage[autostyle, english=american]{csquotes}
\MakeOuterQuote{"}

\usepackage{hyperref}
\hypersetup{
  colorlinks   = true,    % Colours links instead of ugly boxes
  urlcolor     = blue,    % Colour for external hyperlinks
  linkcolor    = blue,    % Colour of internal links
  citecolor    = red      % Colour of citations
}

\definecolor{codegreen}{rgb}{0,0.6,0}
\definecolor{codegray}{rgb}{0.5,0.5,0.5}
\definecolor{codepurple}{rgb}{0.58,0,0.82}
\definecolor{backcolour}{rgb}{0.95,0.95,0.95}

\usepackage{listings}
\lstdefinelanguage{JavaScript}{
  keywords={typeof, new, true, false, catch, function, return, null, catch, yield,
  					    switch, var, if, in, of, for, const, let, while, do, else, case, break},
  ndkeywords={class, export, boolean, throw, implements, import, this},
  sensitive=false,
  comment=[l]{//},
  morecomment=[s]{/*}{*/},
  morestring=[b]',
  morestring=[b]",
  morestring=[b]`
}
\lstdefinelanguage{Pseudocode}{
  keywords={def, for, while, in, if, else, elif, return, case, of, pass},
  ndkeywords={},
  sensitive=false,
  comment=[l]{//},
  morecomment=[s]{/*}{*/},
  morestring=[b]"
}
\lstdefinelanguage{TraceDiff}{
  keywords={Event},
  ndkeywords={COMMON, LEFT, RIGHT},
  sensitive=false,
  morestring=[b]"
}
\lstdefinestyle{mystyle}{
    backgroundcolor=\color{backcolour},   
    commentstyle=\color{codegreen},
    keywordstyle=\color{magenta},
    numberstyle=\tiny\color{codegray},
    stringstyle=\color{codepurple},
    basicstyle=\ttfamily\footnotesize,
    breakatwhitespace=false,         
    breaklines=true,                 
    captionpos=b,                    
    keepspaces=true,                 
    numbers=left,                    
    numbersep=5pt,                  
    showspaces=false,                
    showstringspaces=false,
    showtabs=false,                  
    tabsize=2
}
\lstset{style=mystyle}

\usepackage{csvsimple}
\usepackage{longtable}
\setcounter{LTchunksize}{1}
\usepackage{makecell}

\csvstyle{webisteList}{
before reading=\footnotesize,
late after last line=\\\hline,
respect all=true,
separator=tab,
after reading=\normalsize,
head to column names}

% koniec definicji

\begin{document}
\maketitle

%tu idzie streszczenie na strone poczatkowa
\begin{abstract}
Ads are the main source of income of numerous websites. 
However, some of them are fairly annoying which causes many users to use adblocking browser extensions. 
Some services, in turn, use specialized scripts to detect such plug-ins 
and silently report them or block some content as a punishment. 
The goal of this thesis is to build a pipeline for detecting such scripts based on a differential execution analysis, 
a method provided by other authors in 2018. 
Such a mechanism can be used later to analyze the prevalence of anti-adblockers 
on Polish websites or to build an extension capable of circumventing such scripts.

\end{abstract}

\tableofcontents
%\listoffigures
%\listoftables


\chapter*{Introduction}
\addcontentsline{toc}{chapter}{Introduction}

In the modern-era Internet most web pages operate for profit. Most of them, however, choose to provide free content
in exchange for displaying paid advertisements, or ads for short. Unfortunately, not all websites play fair. Some of them
concentrate on displaying as many ads as possible, and generate traffic by using click-baits 
and other shady practices. Even websites with valuable content can have an overwhelming amount 
of ads. This leads to grave dissatisfaction of some portion of users. To make their browsing experience better,
they turn to ad-blocking extensions.

Operation of the most popular adblockers is based on user-curated lists. They have been around for such a long
time that they became really effective. Not only they are able to block the vast majority of advertisements
but also they rearrange the remaining content to provide seamless experience.
In order to circumvent them, one has to put an ongoing effort to be able
to catch up. It is such a common occurrence that even some browsers have ad-blocking
functionality built-in (e.g. Opera).

The amount of users utilizing ad-blockers cannot be ignored. In 2019 there was over 615 million devices worldwide
with adblocker installed \cite{pagefair:adblock-report}. This, in turn, leads to loss of revenue for many businesses.
To combat this, some of them choose to deploy anti-adblockers, i.e., scripts that detect adblocking extensions
and react in some way. They can generate a visible warnings or even even block the website's content entirely.

The anti-adblockers come in many different fashions. Some of them are simple, custom-made scripts that 
set up an obvious bait and check for reaction typical to adblocking extensions.
They may check if some file was not loaded or if the advertisement banners have been tampered with.
Some solutions perform more than one check. The most advanced ones also employ mechanisms that
make the analysis and reverse-engineering harder. Examples include code minification
or obfuscation involving \emph{eval} function.

Some ad providers, publishers and extension creators recognize users' discontent 
and start initiatives like Acceptable Ads \cite{acceptableads}.
The idea is to have a white list of ads that meet certain criteria such as non-intrusiveness.
Such list is embedded into ad-blocking extensions and enabled by default.
Others just focus on making sure that the ads are displayed and continue to bring profit.
The way to do that is by constantly improving anti-adblocking solutions to stay ahead.

The people behind adblockers have mixed approaches to anti-adblocking warnings.
Some of them choose to let all of them be, others only if they are dismissible (and thus 
less effective from the point of view of the site owners). Naturally, there are also other 
players who develop extensions which block all such warnings, regardless of their
intrusiveness. The whole conflict of interests, opinions and values leads to an arms race.

A regular study of anti-adblocking scripts can help both sides of the barricade. Knowing how such scripts behave
can help create better methods of detecting them. That, in turn, can lead to creation of better blocking tools.
On the other hand, being able to automatically detect some scripts usually means that it is also possible
to block them. As a consequence, studying different detection methods can lead to better anti-adblockers as well.

In 2018 Zhu et al. \cite{DBLP:conf/ndss/ZhuHQSY18} proposed a sophisticated method 
of automated detection of such scripts. This method uses what is called a Differention Execution Analysis. 
The whole approach consists of several steps.
First, at least two JavaScript execution traces have to be collected. An execution trace is a list of all events such
as function enter or leave that occurred during program execution. The first trace is of website's code executed
in an environment without any adblocking extensions. The second one is collected with such a plugin active.
These two traces are later compared and checked if there are any differences between them.
If there are, they can be attributed to anti-adblocking scripts.

There are lots of difficulties to get the whole mechanism working. First of all, trace collection has
to be hand-made. There is no off-the-shelf solution that would meet all requirements of this method.
Second, JavaScript event-based execution model makes it problematic to compare execution traces
as the order and number of events can differ greatly. This issue is solved 
by processes called trace slicing (or untangling) and trace matching. The former is a method of gathering 
execution events into subtraces that correspond to different events. The latter is a process of
pairing subtraces from two different execution traces.
Third, a website can incorporate tons of noise. The web page can utilize some random functions, 
there can be network errors, the content can be generated or selected dynamically.
The authors of the method give a glimpse of how to battle each of the difficulties, 
but leave out a lot of details in most cases. 

\todo{transform into full sentences?}
This work is about implementing a similar system based on the aforementioned idea.
The contributions of the thesis are the following:
\begin{itemize}
  \item The entire method has been implemented \todo{explain what end-to-end means} 
           end-to-end and each step explained in detail.
  \item A novel approach based on Stable Marriage Problem to solve trace matching problem
           has been introduced.
  \item A new approach to filtering execution noises has been introduced and tested.
  \item The entire pipeline has been evaluated and used to conduct a small-scale study on the most
           popular websites in Poland.
  \item An overview of anti-adblocking methods has been presented, based on the pipeline evaluation.
\end{itemize}

First part of the system is Chromium browser with manually instrumented V8 engine that produces
execution traces. The browser is run automatically a few times for each tested website.
Later, the traces are filtered and analyzed to detect execution differences.

The implementation has been manually tested on 100 websites and proved to be a useful tool 
in detecting and analyzing anti-adblocking scripts. 
The analysis of those websites resulted in identification of the most popular mechanisms and solutions.

The study conducted on 300 most popular web pages in Poland revealed that anti-adblocking scripts
are far more widespread than what can be conjectured by just studying visible reactions.


\chapter{Basic concepts}

\section{Definitions}

\begin{itemize}
  \item Execution event -- each occurence of control evaluating some expression, entering or leaving a control statement etc.
  \item Execution trace -- a series of execution events collected during program execution. 
           It is dependent both on program structure and its input (also implicit such as randomly generated numbers).
  \item Positive trace -- an execution trace collected with an active adblocking extension.
  \item Negative trace -- an execution trace collected without an adblocking extension.
  \item Execution index -- a concept formally introduced by Xin et al. \cite{sigplan:execution-indexing}. 
                                         For our purposes we can define it is any function that uniquely identifies execution points. In our case it will
                                         be a statement source map information (file name and precise location of the statement) 
                                         with the current function stack.
\end{itemize}


\section{Adblockers}

Online advertising market is growing each year. The Interactive Advertising Bureau's report for 2018 \cite{iab:2018-report}
states that the Internet advertising revenues surpassed \$100 billion annually in the US in 2018. Moreover, according the same report,
online advertising is the biggest ad media, approximately 30\% bigger than TV.

What it means for the Internet users? That there will be more and more ads. 
Some sources estimate that an average Internet user is servered 11250 ads per month \cite{huff:too-many-ads}.

Not surprisingly, ads are the main source of income for numerous websites. By displaying ads, authors are able to provide
valuable content free of charge. 

However, there are multiple reasons why users may not want to see online advertisements \cite{pagefair:adblock-report}: 
\begin{itemize}
  \item There are websites which sole purpose is to earn money by displaying as many ads 
           as possible without providing any interesting or original content.
           They often generate traffic using click-baits and similar shady practices.
  \item Ads often lenghten pages loading times. The slower the connection, the more annoying it becomes.
  \item Ads increase webpage payload size, which generates higher cost on mobile connections.
  \item Some ads track users, which raises privacy concerns.
  \item Ads can be used to spread malware (in this case it is called Adware) \cite{adblock:adware}.
  \item Some of them are annoying.
  \item They occupy the screen space, leaving less area for the content,
           which can be especially frustrating on devices with small screens.
\end{itemize}

One of the solutions is to use special browser extensions, called ad blockers (or adblockers), 
that prevent ads from being displayed.
Their work is based on community-curated list of filters. Those filters are used to first identify some portions
of website as ads and later remove them.

The removal depends on the type of an ad. Whole page overlays can simply be not shown without disrupting
the website UI. On the other hand, after banners removal, there remains some blank space, which is usually
fixed by repositioning adjacent elements. In some cases, particularly when ads are served from domain
of some ad provider, the requests that download ads can be blocked, thus saving the bandwidth and reducing data usage.

There is some controversy concerning morality of use of such extensions.
One of the most popular ad-blocking extension, uBlock Origin states in its manifesto,
that it is the users' right to have control over what content should be accepted in their browser \cite{ublock:manifesto}.
As if other arguments for using adblockers were not enough of a justification.

Some ad-serving recognize users' frustration and create initiatives like Acceptable Ads \cite{acceptableads}.
In this program, advertisements meeting certain criteria are whitelisted by the commitee.
The whitelist is active by default in adblocking extensions that join the initiative.
Some bigger ad providers are also charged for being whitelisted.


\section{Anti-adblockers}
\label{anti-adblockers}

In 2017 PageFair prepared a report on adblockers usage \cite{pagefair:adblock-report}.
Worldwide, 11\% users use ad blocking solutions. This may seem small, but it is not not uncommon
for some markets to around 20\% or higher penetration (e.g. USA -- 18\%, Germany -- 29\%, Indonesia -- 58\%).

Such high percentage of users not seeing ads means lost income for many businesses.
To recover lost revenue, many websites started deploying anti-adblocking scripts.
Their goal is simple -- when they detect that content blocking extension is present, 
they take some action, potentially mitigating the problem.
The action can vary from simply just reporting the use of extension to the backend to blocking 
the content entirely.

The aforementioned PageFair report studied so-called "adblock walls", i.e. mechanism
preventing users from seeing the website content when they have an adblocker enabled.
The study shows that 90\% of ablockers users have come upon an adblock wall.
More interestingly, 74\% of them leave the website when confronted with such a wall.

Anti-adblockers come in many variants. There are simple, custom scripts written 
specifically for one service, but there are also some sophisticated scripts 
provided by third parties, designed to be easily integrated on any site.

One rather simple example is presented by company offering "Adblock Analytics" service \cite{detect-adblock}.
In their example they add a file named \emph{ads.js} with a short JavaScript code that adds a hidden div block with an unique id.
Ad-blocking extensions usually block files named like that. All that remains to detect the extension is to
check whether the div element was indeed added to the DOM tree. If not, it is a sign that an adblocker is active.

Another example is the "BlockAdBlock" module \cite{github:blockadblock}.
Similarly to the first first example, it first sets up a bait,
i.e. a component that should look like an ad to the adblocker, 
but then the detection phase is a little bit more thorough. It runs the check several times
and it inspects several properties of the bait to make sure it has not been tampered with.
If something changed -- an adblocker is detected.
It also allows to register callbacks for both outcomes of the analysis.

Even people that base their websites on content managements systems have an easy access to such
scripts. Most notably, around 1 in 3 of the top 10 million most popular website runs on WordPress \cite{wiki:wordpress}
and in this case there are multiple extensions available, varying in price and capabilities \cite{wordpress:antiadblockers}.

Naturally, there are multiple users that do not want to disable their adblocker to see the website's content.
As a reaction, there are several solutions developed to combat anti-adblockers.
Usually, they focus on providing new filtering lists, that can be used by the existing adblocking extensions
\cite{anti-adblock-killer, nano-defender}

Unfortunately, those solutions have rather poor effectiveness, as reported by Zhu et al. \cite{DBLP:conf/ndss/ZhuHQSY18}.
This inability to effectively defuse anti-blockers was their most important reason to develop a method of detecting such scripts.


\section{Differential Execution Analysis}

Differential Execution Analysis is a dynamic program analysis method. Its goal is to pinpoint exact differences
in two executions of the same program with different inputs or other conditions (e.g. network or memory errors).
Good overview of the method is presented by Johnson et al. \cite{ieee:alignment-and-slicing}

The analysis is usually performed by collecting an execution trace for each run and then comparing
them using trace alignment. Trace alignment is a process of identifying which fragments of execution traces
are common, where they diverge and when they converge again. The exact algorithm is discussed in section \ref{trace-alignment}.

The results can be useful in various scenarios, i.e. debugging a program or during security analysis.


\section{Detecting anti-adblockers}

Zhu et al. in their paper "Measuring and Disrupting Anti-Adblockers Using Differential Execution Analysis" 
 \cite{DBLP:conf/ndss/ZhuHQSY18} introduced a novel method of detecting anti-adblockers
using Differential Execution Analysis. The work presented in this thesis is based on that method. 
The differences and new ideas are explained in the later sections.

The premise is quite simple. They collect execution traces of JavaScript code on given website
first without any content blocking and then with an adblocker turned on.
Afterwards, they analyze such traces using Differential Execution Analysis. Any differences in execution in both cases
are attributed to anti-adblocking activities of the website.

While the idea is pretty straightforward, the are multiple challenges here. 
First, trace collection is not a trivial task, especially when the interest exceeds just function entry and exit events.
The authors instrument V8 engine to achieve the task, but do not provide much details, apart from briefly stating that
their instrumentation is embedded into native code generation process.

Second, due to JavaScript execution model, described in detail in section \ref{js-exec-model}, execution traces of different
events are interleaved. To battle this issue, traces have to be sliced into subtraces and analyzed pairwise.
In a language with a simpler execution model this step would be unnecessary.
Authors also do not explain their method of how to pair the subtraces. They just mention that all $m \times n$ pairs 
have to be analyzed.

Lastly, the biggest challenge is to combat execution noises, e.g. page randomness or variable content.
Authors solve the issue by loading the same page three times with an adblocker and three times without and
use redundant traces to generate a black list of execution differences.


\section{JavaScript execution model}
\label{js-exec-model}

JavaScript concurrency model is based on an "event loop" \cite{mozilla:event-loop}. The engine is essentially single-threaded
and concurrency is implemented by utilizing a message queue. This queue processes events one by one, to completion, i.e.
a function corresponding to the message starts with a new, empty stack and its processing is done when the stack becomes empty.

The easiest way to add new events to the queue is by calling \emph{setTimeout} or \emph{setInterval}.
Furthermore, all callbacks attached to DOM events (e.g. \emph{onClick}) are executed by adding an event to the queue.

It is worth noting that execution of functions can be intertwined, e.g. when generators are used.
This is the reason why trace slicing is needed.

Each iframe and browser tab has its own execution environment that include a separate message loop, 
more on that in section \ref{v8-in-chrome}.


\chapter{Trace collection}

\section{Methods overview}
There are a few distinct ways to obtain execution trace of JavaScript code. 

The simplest (and most limited) methods use only mechanisms present in the language.
More elaborate inject special tracing code to the analyzed script. The last kind modify the engine to produce
the desired data.

\section{Dynamic in-JavaScript code injection}
JavaScript is a very dynamic language. For this reason, it is relatively easy to write code that will
modify each function present in the environment to log each entry and exit,
possibly along with all the arguments and return value \cite{stack:js-console-log}.

Listing \ref{js-instrumentation} is an example of a code that instruments all functions in a selected object
to log their name and arguments when they are called.
Function \emph{instrument} simply goes through all properties of an object and 
replaces each function with a new function that first logs function
name and all provided arguments and then calls the original function.
This function can be easily extended to also log return value and instrument all subobjects recursively.

However, this is not enough for the needs of Differential Execution Analysis. 
The most obvious limitation is that it is not possible to instrument control statements.
Another major shortcoming is that function can be instrumented only after they are defined. 
It is quite an obstacle, because JavaScript allows to define function practically anywhere and it is very common
to use anonymous functions as callbacks. It is not possible to instrument such callbacks 
without modifying the instrumented code, which brings us the static injection methods.

\lstinputlisting[language=JavaScript, caption=Dynamic instrumentation in JavaScript, label=js-instrumentation]{js/instrument.js}

\section{Static code injection}
Less limited approach is to statically rewrite the instrumented code and inject tracing code wherever it is needed.
The upside of such approach is that there are several ready to use frameworks.
The downsides will be pointed out when discussing each solution.

\subsection{Web Tracing Framework}
One notable solution is Web Tracing Framework developed by Google \cite{google:wtf}.
The main use of this framework is to profile web applications to find performance bottlenecks.
The functionality is similar to that of \emph{Performance} tab in Chromium developer tools.

Notwithstanding, one of its advanced features is closer to our needs. 
It allows the user to first instrument JavaScript sources and then 
collect execution traces and inspect them in a special app.

Having to instrument all source code is cumbersome, especially when we try to analyze code on some
arbitrary websites. For this reason Web Tracing Framework also offers an extension and proxy server 
that cooperate to instrument all JavaScript code online, when it is loaded into the browser.

Unfortunately, this solution has a few deal-breaking downsides:
\begin{itemize}
  \item It logs only function entry and exit events.
  \item Logging format is not public
  \item It is a bit dated, new JavaScript features may not be traced properly
  \item Function defined using \emph{eval} or \emph{Function} will not be traced
\end{itemize}

\subsection{Iroh}
Iroh \cite{iroh} is the most complete solution based on static code injection.
Just like Web Tracing Framework, Iroh also needs to patch the code first, but its capabilities go well
beyond what the previous solution offers. It allows the user to register arbitrary callbacks to practically any
element of JavaScript's Abstract Syntax Tree (AST). It means that this tool is able to instrument \emph{if} statements.
This use case is even included in the official examples.

Unfortunately, the framework does not offer proxy that could instrument code loaded into browser on the fly.
There is also another, more general concern -- performance of such solution may not be acceptable.

\section{Engine instrumentation}
The last option is to modify the JavaScript engine itself to produce execution traces.
The most striking benefit is that the engine already has all required info and the solution 
does not require the analyzed code to be modified.
Another advantage is the performance. Implementing tracing code directly in the engine means 
that there is less indirection. The code does not need to be interpreted by the engine, it is a native code
that is called from JavaScript.

Unfortunately, such instrumentation has to be written almost from scratch. 
Nevertheless, due to the most flexibility and performance advantages, this solution has been chosen 
for this implementation.

The same choice has been made by Zhu et al. \cite{DBLP:conf/ndss/ZhuHQSY18} for their implementation, 
but they did not share their code.

More details on how to instrument the engine in chapter \ref{v8-instrumentation}.


\chapter{Trace collection by V8 instrumentation}
\label{v8-instrumentation}

\section{V8 architecture}
Most modern browsers do not implement JavaScript interpreter directly. Rather, they utilize a more
specialized program called JavaScript engine, which they usually embed. 

V8 is an engine used by the most popular browser, Chrome \cite{v8:main-page}, which in June 2019 
had over 80\% market share  \cite{w3:browsers}.

Let us start with introducing some V8-specific glossary \cite{v8:bindings}:
\begin{itemize}
  \item Isolate -- an instance of V8. There can be more than one Isolate used by one process of embedding application.
  \item Context -- a concept of global variable scope in V8. Each Context has its own global variables and prototype chain.
           Each iframe has its own separate Context. There can be multiple Context in one Isolate, but due to site isolation 
           (see section \ref{v8-in-chrome}), each iframe is run in another process and has its own Isolate.
\end{itemize}

V8 processes JavaScript code in several steps. In this thesis we will focus on steps
directly related to implementing trace collection.

In short, JS code is first parsed into AST, which contatins source map information. 
In the next step V8 traverses the entire AST and emits bytecode for each node.
The bytecode is V8-specific and reflects the architecture of V8's abstract machine.
More on that in section \ref{v8-bytecode}.

It is worth noting that while user-defined function are translated to bytecode,
most built-in functions are implemented in a different way. We will take a closer look at them
in section \ref{v8-builtins}.

Only after the code is translated into bytecode, it is finally executed. At this stage there are two
kinds of functions. First -- those defined in JavaScript, represented in bytecode, second -- builtins
defined in other ways and already compiled into native code. This distinction is not important 
to the user, as those functions do not differ in JavaScript, and can easily call each other.
It it will become important once we try adding instrumentation code.

At some point during execution, functions that are called very often with the same argument types, 
can be compiled into native code by its TurboFan Just In Time compiler \cite{v8:turbofan-jit}.
If later the same function is called with some argument of some other type, it simply gets deoptimized into bytecode.

The engine's architecture is focused on achieving superior performance, while conforming to all
standards and not jeopardizing security. 


\subsection{JS bytecode}
\label{v8-bytecode}
V8's interpreter, Ignition, is a register machine with an accumulator register \cite{medium:js-bytecode}.
While all other registers have to be specified when used as arguments, accumulating register
is implicit, it is not specified by bytecodes that use it.

This section is supposed to be only a shallow dive into V8's bytecode. We will have a look at one simple example
to be able to understand what is going on in section \ref{v8-bytecode-injection}.

Let us have a look at a simple JavaScript function and see the bytecode produced by Ignition.
\footnote{V8 prints out bytecode when flag \emph{-{}-print-bytecode} is provided}
Listing \ref{js-factorial} shows a naive implementation of a function calculating factorial.
It includes function call (line 5) because the interpreter is lazy and otherwise the function would not be compiled.
List \ref{bytecode-factorial} presents bytecode generated by V8's interpreter for that function.

\lstinputlisting[language=JavaScript, caption=Calculating factorial in JavaScript, label=js-factorial]{js/factorial.js}
\lstinputlisting[caption=Ignition bytecode for function \emph{factorial}, label=bytecode-factorial]{bytecode/factorial.b}

The listing starts with the parameter count, register count and frame size. The first one may be baffling at first
since \emph{factorial} takes only one number as an argument. We have to remember 
that all JavaScript functions also take implicit arguments \emph{this}.

After the header, the actual bytecode is listed. The first number and letter, e.g. "18 E" in line 5 identifies expression (E)
or statement from the source file. The number is an offset in characters.
It is followed by the code's address in memory, offset (in bytes) from function start and the code itself in a hexadecimal form.
All of them are rather useless for us. The most useful parts are the codes in human-readable form at the end of each line.

Once we recall that most codes use accumulating register, reading the code becomes relatively self-explanatory.
To make things easier, the use of accumulating register is reflected in the code's name, e.g. 
\emph{Ld\textbf{a}Constant [0]} loads constant numbered 0 to the accumulating register.

We should now be ready to interpret each line of \emph{factorial}'s bytecode.
Upon function entry (line 5) the validity of the stack is checked. Later, integer 1 is loaded into
accumulating register. Next, argument 0 (symbol $a0$), which happens to be $n$, is tested 
to be less that or equal to the value stored in the accumulating register (1).
If not, the jump (to line 11 in the listing) is performed. If the conditional was true, integer 1 is loaded
into accumulating register, then the control jumps to the last instruction which returns from the function.
The return value is always stored in the accumulating register, so 1 is returned.
If the conditional was true, and we are in line 11, constant 0 is loaded, which happens to be the name of our function.
Later, that name is stored in register $r1$, argument 0 is loaded into the accumulating register, 1 is subtracted and the result
is stored in register $r2$. Next, function of name stored in $r1$ (\emph{factorial}) is called with value stored in 
register $r2$ ($n-1$). The result of the call, stored in the accumulating register, is then multiplied by the first argument
($n$). The result of multiplication is already in the accumulating register and the function can now return.


\subsection{JS built-in functions}
\label{v8-builtins}

According to V8's documentation post \cite{v8:built-ins}, JavaScript built-in functions can be implemented
in three different ways. They can be written in JavaScript directly, implemented in C++ (runtime functions)
or defined using an abstraction called Code Stub Assembly (CSA). The post, however, is slighty dated. Since then,
new abstraction, Torque \cite{v8:torque}, has been added.

It is not important to know how to write CSA or Torque code. It is only important to remember 
that most built-in functions are implemented in a different way than a user-defined ones.


\section{V8 usage in Chromium}
\label{v8-in-chrome}

Today's usage of V8 in Chromium is determined in large part by security concerns \cite{v8:spectre}.
For us, the most important decision was made after discovery of Spectre \cite{Kocher2018spectre} 
and Meltdown \cite{Lipp2018meltdown} vulnerabilities.
To increase protection against attacks based on those two vulnerabilites, Chrome's team decided to make
Site Isolation enabled by default. \cite{chrome:site-isolation}

Each website can have multiple iframes -- the default one and some embedded ones. All of them
are run in a separate process and, as a consequence, have their own instances (Isolates) of V8.


\section{Chrome's extensions architecture}

In the current model, Chrome's extensions may consist of the following components \cite{chrome:extensions}:
\begin{itemize}
  \item Manifest -- a file describing an extension, listing all its files and capabilities.
  \item UI Elements -- code adding extension's user interface.
  \item Options Page -- a page allowing the user to customize the extension.
  \item Background script -- a file with callbacks for browser events. Run only when an event with a registered callback occurs.
           It runs in its own Context, in a separate process
  \item Content script -- extension's code that is run in the page's Context. This code can read and modify DOM
           of the website and communicate with partent extension via messages or storage.
           It can also access a limited subset of Chrome's APIs directly, mostly those needed to communicate
           with parent extension \cite{chrome:content-scripts}.
\end{itemize}


\section{V8's \emph{-{}-trace} flag}

Usually the easiest way to implement some new functionality is to find a code that provides a similar
functionality and extend it. In case of tracing, such base is provided by the V8's \emph{-{}-trace} flag.

First, we will inspect what this flag can do. The example from the section \ref{v8-bytecode} (Listing \ref{js-factorial}) will be reused.
Listing shows console output of V8 with \emph{-{}-trace} flag enabled. The output is well-formated and self-explanatory.
Unfortunately, it has some shortcomings. First, there is no source map info. Second, due to the nature of JavaScript,
function traces can get intertwined (more on this in section \ref{js-exec-model}. And since stack information is limited to
just the stack depth, it may be impossible to untangle events in some cases.

\lstinputlisting[caption=V8's output for \emph{factorial} with \emph{-{}-trace} flag, label=trace-factorial]{out/factorial-trace.o}

Nevertheless, this flag is a good starting point. Let's have a deeper dive into how it works.

We have already seen the bytecode for \emph{factorial} in listing \ref{bytecode-factorial}.
Listing \ref{bytecode-factorial-trace} shows the bytecode for the same function when \emph{-{}-trace}
flag is enabled. There are two differences compared to the bytecode produced without tracing flag.
First -- there is a call to runtime function \emph{TraceEnter} before \emph{StackCheck} upon function entry. 
Second, just before returning, the result is saved to register $r0$ and runtime function \emph{TraceExit} is called
with return value as its argument. \emph{TraceExit} stores its argument back in accumulating register so there
is no change in semantics of the inspected code.

\lstinputlisting[caption=Ignition bytecode for function \emph{factorial} with \emph{-{}-trace} enabled, 
	label=bytecode-factorial-trace]{bytecode/factorial-trace.b}


\section{Bytecode injection}
\label{v8-bytecode-injection}

Finally, we have come to the gist of the current chapter -- tracing implementation.
Once we have seen how \emph{-{}-trace} flag works, we can improve it to our needs.

The entire implementation requires a few steps:
\begin{enumerate}
  \item Adding two new flags -- one for turning on our tracing (\emph{-{}-trace-dea}), and one for specyfing the file with tracing info
  \item Preparing a function that prints out the entire stack with source map information
  \item Preparing a set of new runtime functions, one for each type of event we want to trace
  \item Injecting calls to runtime functions in the appropriate places
\end{enumerate}

We will not delve too deeply on how to implement each part.
Points 1 and 3 are pretty straightforward. Both of them just require adding declaration to special header files.

Point 2 seems the hardest, but luckily there is a similar function in the Chromium codebase.
It is worth noting that it is possible to recover source map information during runtime
and have the entries contain precise line and column info. However, it turned out to be too slow
for use in Chromium, as there was a noticeable slowdown. Almost no webpage could finish loading in a reasonable time.
The reason for such poor performance is that those locations are not stored directly in AST. Rather, only offset
in characters is stored. To recover line and column info it is necessary to first go through
a few levels of abstraction to get the source code and then calculate the coordinates by traversing it.
As a consequence, only character offset is displayed. It is always possible to recover more friendly
line, column location later, so it is not that big of a deal.

The challenge of part 4 is to have the right offsets of statements/expressions.
It is the easiest with functions, as location of beginning of their definition is stored in an 
object representing the function during runtime. In case of usual statements it is harder 
as their locations are available only during parsing and bytecode generation stages.
To have those offsets accessible during runtime, they are stored as code constants
and passed as arguments to runtime functions that print out the code events.

Listing \ref{bytecode-factorial-trace-dea} shows bytecode generated by Ingition with
our tracing flag enabled. Similarly to the original tracing, there are calls to runtime functions
upon entry (line 5) and just before returning (lines 23-24). 
The new part is that also the information which branch was taken is logged.
In lines 7-8 the offset information is stored in register $r0$ which is later used by runtime
functions that perform the actual logging (lines 12, 15)

\lstinputlisting[caption=Ignition bytecode for function \emph{factorial} with \emph{-{}-trace-dea} enabled, 
	label=bytecode-factorial-trace-dea]{bytecode/factorial-trace-dea.b}
		
Listing \ref{trace-factorial-dea} shows the output produced by the new flag. Each execution event
entry consists of event type, location (optional, depends on event type) and full call stack.
Events returning value also log that value, but it is not used later in the pipeline.
Stack is represented as a list of locations. Each location consists
of function name, file of origin and offset in the file (in characters). The $-1$ that appers after
the offset is a remnant of the implementation that recovers full position info (line, column).
The next part of the pipeline expects two numbers here. This way it is easy to turn on full position recovery
and have the pipeline still working.

Execution events that are logged in the current implementation:
\begin{itemize}
  \item Function enter and exit
  \item Generator enter, suspend, yield (exit is indistinguishable from a normal function exit)
  \item If statement then/else paths
  \item Ternary expression truthy/falsy value (same as if statement paths)
\end{itemize}

It is easy to extend the implementation to also log execution events associated with loops, but there
is a tradeoff between coverage and size of log files created, so it was not done.
	
\lstinputlisting[caption=V8's output for \emph{factorial} with \emph{-{}-trace-dea} flag, label=trace-factorial-dea]{out/factorial-trace-dea.o}

Careful reader might have noticed that in listing \ref{js-factorial} in the last line there is a call to \emph{factorial}
and the result is passed to \emph{console.log}, but only the former call is logged.
The reason for this is that the implemented solution works only for functions defined in JavaScript
(it is also true for the default \emph{-{}-trace} flag)
Functions defined in other ways (see section \ref{v8-builtins}) are not logged. It is certainly possible
to add logging to each one of them by modyfing their code or by modifying CSA/Torque compiler,
but it has not been done here. The amount of work required to instrument also those function
seems disproportionate to the potential improvements. Another justification is that they are
a black box anyway, there is no JavaScript code corresponding to them 
and we cannot see branch divergences happening inside them.

The last obstacle worth noting is the Chrome's process separation. As explained in section \ref{v8-in-chrome}
there are multiple instances of V8 running at the same time, in different processes.
When some flag is passed to Chrome, all instances see the same value. Therefore, if some file is passed,
all processes will write to the same file, possibly resulting in interlacing and unusable output. 
An easy workaround is to create a new file for each Isolate and later just select only the interesting one 
(the one that corresponds to the analyzed website). There will be always at most one such file 
(theoretically a website could not contain any JavaScript code) and it is easy to select it by greping 
by source location. All other files can be deleted.

V8 Isolate is generally pretty oblivious to what code it runs. Website code is the same to V8 
as extension code. After all, the environment (Web API, DOM, etc.) is provided by the browser.
Nevertheless, it is preferable not to log adblocker events. Such extensions have really long lists of filters
and their intialization takes 1-2 seconds when the browser is not producing execution traces.
When it does, the initialization takes more than half a minute on a slow computer and the trace itself 
occupies 4.5GB. Again, we can resort to a trick: inspect the printed event and stop tracing in a given 
Isolate when some extension-specific file has been encountered. As a result, extension code is not
traced and it has no noticeable performace deterioration.


\section{Controlling Chrome programatically}

Adblocking extensions need a second or two to initialize their code. Even in a stock browser,
opening a website immediately after the browser starts can result in ads not being blocked.
As a consequence, simply passing the website address through console is not enough 
when we want to collect traces automatically. The extension will simply not be fully initialized
and website will not be blocked. Some more sophisticated solution is needed.

Fortunately, there is a framework for building end-to-end tests -- Selenium \cite{selenium-python}. 
It has binding in all major languages, in our case we will use Python.

Selenium is capable of doing any interaction with the website that user can do.
It communicates directly with a specialized WebDriver, different for each browser.
In case of Chrome it is called ChromeDriver. WebDriver issues the command to the browser through
the debugging interface, which is a special port with well-defined communication protocol.
And that is all we know about it, as we are not using any of its sophisticated features, 
just connecting to the browser, opening a website and closing it after some time.


\chapter{Trace analysis}

\section{Parsing}
Although the format presented in chapter \ref{v8-instrumentation} is really simple, parsing it can prove to be challenging.
The sole reason is the size of the files. It is really common for websites to produce files of size in the range of a few Gigabytes.
Some can even output as much as 48 GB in about 2 minutes!

The analyzing program was written in Haskell \cite{haskell:main-page}.
Haskell is a purely functional, lazy, statically typed language.
It has been chosen because it is relatively easy to write parsers in a language like this.
Further, static typing with pattern matching proves convenient when manipulating well-structured data like log entries.
Last but not least, automatically derived instances
\footnote{Without going into details, classes in Haskell are a bit like interfaces in 
object-oriented languages, e.g. class \emph{Ord} defines objects that can be compared}
help avoid writing tedious code and focus on implementing non-trivial parts.

All that being said, Haskell is a bit like C++ -- inexperienced used can easily make grave mistakes
that render the code slow and memory-greedy.

The logging format was described in section \ref{v8-bytecode-injection}.
Each entry consist of event type, and several location. Each location comprises
function name, source file and position in the file.

At first, internal format for the event was exactly the same. One object consisted of
two strings and two numbers (it could be one number, but this way line and column info
logging can be turned on any time).

The first Haskell-based mistake was to use \emph{String} types to represent function and file
names. The use of default \emph{String} to represent textual data makes the code inefficient 
because it is a linked lists of \emph{Char}s, i.e. to store one character 9 bytes of memory are used.
It is so wrong and inefficient to use this type that it is not worth trying to profile such implementation.

That error was fixed by changing the representation to \emph{ByteString} from \emph{bytesting} 
package \cite{haskell:bytestring}, which 
is a class of immutable byte arrays (perfect for ASCII-based texts). 
This change was accompanied by rewrite of the parsing code to \emph{attoparsec} \cite{haskell:attoparsec}.

The code was now much less memory-consuming and faster, but it still seemed to consume too much memory.
Profiling proved that suspicions were warranted.

\todo{Insert profiling graph}

Pinned memory seemed to be never freed during the execution. Also, traces seemed to reside in memory
for too long and occupy too much space. 

First problem was a reflection of how Haskell keeps \emph{ByteString}s.
They are stored in pinned memory, which is kept on a heap but 
not managed by garbage collector \cite{haskell:shortbytestring-and-text}. 
As a result, it leads to fragmentation of the heap and cannot be effectively managed.
The fix turned out to be simple. The is more suitable storage format -- \emph{ShortByteString} 
(also part of the \emph{bytestring} package). It is managed
by GC and kept just usual heap object, so the memory can be better managed.
\footnote{So why even use \emph{ByteString}? Because it stored in pinned memory, it can be passed to foreign functions.
Also, it is more efficient when the strings are really long}

The second problem was harder to track down. This time the reason was the laziness.
Laziness can improve the perfomance when some objects are never used and the language never completes
computing them. However, when we know that all objects will eventually be used, it is more efficient
to calculate them as soon as possible. The enforcement of eager evaluation seems to be an obscure
feature, but when we know what is going on, it is fine.
The improvement was visible, but it was not the end.

\todo{Profiling graph}

It seems suspicious that the entire file has to be kept in memory to be parsed. After all, if it was written in C++,
probably one line at a time would be read and processed. So why this parser consumes so much memory?

This is what \emph{attoparsec} states about incremental input:

\begin{displayquote}
Note: incremental input does not imply that attoparsec will release portions of its internal state for 
garbage collection as it proceeds. Its internal representation is equivalent to a single ByteString: 
if you feed incremental input to a parser, it will require memory proportional to the amount of input you supply. 
(This is necessary to support arbitrary backtracking.)
\end{displayquote}

So, our parser keeps the whole file contents in memory to be able to backtrack. But it is not necessary with
such a simple format. Fortunately, it is possible to improve this behaviour.
Currently, the parser tries to parse the entire file into list of events. To be sure that it can backtrack,
it keeps the whole input in memory. Instead, we can ask the parser to parse only one event. 
It will do it and return the part of the input that was not parsed yet. We can repeat that in a loop
and parser will never keep more memory that just a few kilobytes.

Last, but not least, the internal representation of the event can be greatly improved.
Locations consist of file names and function names. But both sets are limited!
Usualy there are only a few sources containing just a few hundreds of unique functions. 
We can create a map of all source and function names and just keep appropriate ids in location objects.
Just 4 numbers instead of 2 strings and 2 numbers!

This representation also has another major benefit -- comparisons of such objects are much fasters.

The final memory consumption is presented in figure

\todo{Profiling graph}

\section{Trace untangling}
Due to the nature of JavaScript execution model (see section \ref{js-exec-model}), execution events
corresponding to different JavaScript events or functions can be intertwined.

For this reason, all events forming a trace have to be untangled into subtraces, each corresponding
to one script or callback. 

Let us recall that all events are logged by our instumented Chromium with their call stacks.
Now, we have two cases:
\begin{itemize}
  \item The event is a function entry with an empty call stack -- it starts a new subtrace
  \item The event is a function exit removing the last item from the stack -- it ends a subtrace
  \item Any other event -- it is a continuation of a subtrace, whose last event has the same call stack
\end{itemize}

Naturally, some care has to be taken to ensure that the right stack is compared. Some events change the
call stack, e.g. function enter and exit, some do not, e.g. "if statement -- then".
Here, the stack \emph{after} the event is always saved and it is appopriately modified 
(one location can be added, removed or nothing is changed) when comparing
with past events' stacks.

Listing \ref{alg-untangling} presents the pseudocode of the algorithm.

\lstinputlisting[language=Pseudocode, caption=Trace untangling, label=alg-untangling, mathescape=true]{algorithms/untangling.alg}

The above code is rather uncomplicated, but it is slow when implemented naively.
The most wasteful part is finding the trace with the right stack in \emph{findMatchingSubtrace}.
To make it faster, two things were done:
\begin{itemize}
  \item Location objects are just 4 numbers instead of 2 strings and 2 numbers. This makes stack comparisons
           one or two orders of magnitude faster
  \item Open traces are kept in a map indexed by stack of the last event. 
  			The lookup becomes logarithmic instead of linear. This optimization is especially important when there are lots
  			of intertwined subtraces. It also must be noted that this optimization alone wouldn't help much if the comparisons 
  			between stack were slow
\end{itemize} 

\section{Trace alignment}
\label{trace-alignment}

\cite{ieee:alignment-and-slicing}

\section{Trace matching using SMP}

\section{Noise filtering}




\chapter{Evaluation}
\section{Evaluated websites}
\section{Detected anti-adblockers}

%\lstlistoflistings
%\addcontentsline{toc}{chapter}{Listings}

%\listoffigures
%\addcontentsline{toc}{chapter}{List of Figures}

%\listoftables
%\addcontentsline{toc}{chapter}{List of Tables}

\cleardoublepage
\phantomsection
\addcontentsline{toc}{chapter}{Bibliography}
\bibliography{sources}{}
\bibliographystyle{plain}

\chapter*{Appendix A}
\addcontentsline{toc}{chapter}{Appendix A}

\csvreader[webisteList,
  longtable= r | p{12cm} | c | r ,
  table head=\hline & URL & \thead{Vis. \\ warn.} & \thead{Diff. \\ count}\\\hline
]{tsv/alexa.tsv}{}%
{\thecsvrow & \url{\adr} & \wall & \cnt}%

\end{document}


%%% Local Variables:
%%% mode: latex
%%% TeX-master: t
%%% coding: latin-2
%%% End:
