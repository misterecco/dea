\chapter{Trace collection}

In this chapter we discuss available methods of collecting execution traces of JavaScript code.

There are a few distinct ways to obtain an execution trace. 
The simplest (and the most limited) methods use only mechanisms present in the language.
Others inject special tracing code to the analyzed script. The most profound ones modify the engine to produce
the desired data.

\section{Dynamic in-JavaScript code injection}
JavaScript is a very dynamic language. For this reason, it is relatively easy to write code that will
modify each function present in the environment to log each entry and exit,
possibly along with all the arguments and the returned value \cite{stack:js-console-log}.

Listing \ref{js-instrumentation} presents a code example that instruments all functions in a selected object
to log their name and arguments when they are called.
Function \emph{instrument} goes through all properties of an object and 
replaces each member function with a new function that first logs the function
name and all provided arguments and then calls the original function. This function can be easily extended to also log the returned value and to instrument all subobjects recursively.

However, this is not enough for the needs of Differential Execution Analysis. 
The most obvious limitation is that it is not possible to instrument control statements.
Another major shortcoming is that functions can be instrumented only after they are defined. 
It is quite an obstacle, because JavaScript allows to define function practically anywhere and it is very common
to use anonymous functions as callbacks. It is not possible to instrument such callbacks 
without modifying the instrumented code which brings us to the static injection methods.

\lstinputlisting[language=JavaScript, caption=Dynamic instrumentation in JavaScript, label=js-instrumentation]{js/instrument.js}

\section{Static code injection}
\label{static-injection}
A less limited technique is to statically rewrite the instrumented code and inject tracing code wherever it is needed.
The upside of such an approach is that there are several ready-to-use frameworks.
The downsides are pointed out below, for each solution separately.

\subsection{Web Tracing Framework}
One notable system is the Web Tracing Framework developed by Google \cite{google:wtf}.
The main use of this framework is to profile web applications to find performance bottlenecks.
The functionality is similar to that of the \emph{Performance} tab in Chromium developer tools.

Notwithstanding, one of its advanced features is closer to our needs. 
It allows the user to instrument JavaScript sources 
to collect execution traces and inspect them in a special application.

Having to instrument all source code is cumbersome, especially when we try to analyze the code on some
arbitrary website. For this reason, the Web Tracing Framework also offers a browser extension and a proxy server 
that cooperate together to instrument all JavaScript code online, when it is loaded into the browser.

Unfortunately, this solution has a few deal breaking downsides:
\begin{itemize}
  \item It logs only function entry and exit events.
  \item The logging format is not public.
  \item It is a bit dated, new JavaScript features may not be traced properly.
  \item Functions defined using \emph{eval} or \emph{Function} constructor are not traced.
\end{itemize}

\subsection{Iroh}
Iroh \cite{iroh} is the most complete solution based on static code injection.
Just like the Web Tracing Framework, Iroh also needs to patch the code first but its capabilities go well
beyond what the previous solution offers. It allows the user to register arbitrary callbacks to practically any
element of JavaScript's Abstract Syntax Tree (AST). It means that this tool is able to instrument \emph{if} statements.
This use case is even included in the official examples.

Unfortunately, the framework does not offer a proxy server that could instrument code loaded into browser on the fly.
There is also another, more general concern -- the performance of such solution may not be acceptable.

\section{Engine instrumentation}
The last option is to modify the JavaScript engine itself to produce execution traces.
The most striking benefit is that the engine already has all required info and the solution 
does not require the analyzed code to be modified.
Another advantage is the performance. Implementing tracing code directly in the engine means 
that there is less overhead. The code does not need to be interpreted by the engine, it is native code
that is called from JavaScript.

Unfortunately, such an instrumentation has to be written almost from scratch. 
Nevertheless, due to its flexibility and performance advantages, this solution has been chosen 
for this implementation.

The same choice has been made by Zhu et al. \cite{DBLP:conf/ndss/ZhuHQSY18} for their implementation, 
but they did not share their code.

More details on how to instrument the engine can be found in Chapter \ref{v8-instrumentation}.
